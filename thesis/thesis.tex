% This thesis template is based on the personal template of Felix Moebius. Thanks, Felix!
% It has been extended and modified by Tobias Pfandzelter at the Scalable Software Systems group of TU Berlin.
% It is licensed under the terms of the MIT license, meaning you are free to use it however you see fit but we accept no liability.
% Good luck writing your thesis!

\documentclass[a4paper, 11pt]{article}

% \usepackage[utf8x]{inputenc}
% \usepackage[T1]{fontenc}
\usepackage{fix-cm}
\usepackage{fontspec}

\usepackage[a4paper, margin=3cm]{geometry}
\usepackage[titletoc, title]{appendix}

\usepackage{tikz}
\usetikzlibrary{shapes,arrows,positioning,fit,backgrounds,calc}

\usepackage{color}
\usepackage{booktabs}
\usepackage[all]{nowidow}
% \usepackage[dvipsnames]{xcolor}
\usepackage[hidelinks]{hyperref}
\usepackage{acronym}
\usepackage{graphicx}
\usepackage{url}
\usepackage{titlesec}
\usepackage{csquotes}

\usepackage{transparent}
\usepackage{eso-pic}
\usepackage[section]{placeins}
\usepackage{setspace}
\usepackage{parskip}
\usepackage{subcaption}

\usepackage{array}

% auto adjusting width tables
\usepackage{tabularx}
\newcolumntype{Y}{>{\centering\arraybackslash\hspace{0pt}}X}

\renewcommand\thefigure{%
\thesection.\arabic{figure}}
\renewcommand\thesubfigure{%
\thesection.\arabic{figure}.\arabic{subfigure}}
\renewcommand\thetable{%
\thesection.\arabic{table}}

\usepackage[main=english, ngerman]{babel}

% we use the cleveref package to refer to figures, sections, etc.
% instead of "Figure~\ref{fig:example}", write only "\cref{fig:example}" and the word "Figure" (or table, etc) will be inserted normally
\usepackage[noabbrev,capitalise]{cleveref}

\usepackage[
  maxbibnames=99,
  style=alphabetic,
  url=false,
  backend=biber,
  sortcites=true,
]{biblatex}
\addbibresource{refs.bib}
\DeclareFieldFormat[online]{urldate}{Last accessed: #1}
\DeclareFieldFormat{eprint}{arXiv: \href{https://arxiv.org/abs/#1}{#1}}
\DeclareFieldFormat[report]{title}{``#1''}

\newcommand{\projectTitle}{Benchmarking the Scalability Behavior of MobilityDB and CrateDB}
\newcommand{\thesisType}{Bachelor's Thesis}
\newcommand{\authors}{Eryk Karol Kściuczyk}
\newcommand{\matrikel}{468796}
\newcommand{\authorEmail}{\href{mailto:e.ksciuczyk@campus.tu-berlin.de}{e.ksciuczyk@campus.tu-berlin.de}}
\newcommand{\examinera}{Prof.~Dr.-Ing.~David Bermbach}
\newcommand{\examinerb}{Prof.~Dr.~habil.~Odej Kao}
\newcommand{\supervisor}{Tim Christian Rese}

\newcommand{\projectYear}{2025}
\newcommand{\facultyName}{Fakultät Elektrotechnik und Informatik}
\newcommand{\departmentName}{Fachgebiet Scalable Software Systems}

\newcommand{\mobilitydbc}{MDBC~} % Other option is MobilityDB-C
\newcolumntype{P}[1]{>{\centering\arraybackslash}p{#1}}

\begin{document}
\input{front.tex}

\newpage

\section*{Abstract}
Spatiotemporal databases increase in demand as smart cities emerge and a large horizontally scalable systems are needed to handle IoT workloads.
In this thesis we benchmark and compare the horizontal scalability of two open-source distributed database management systems (DDBMS): MobilityDB extended with Citus for horizontal scalability (MobilityDBC) and CrateDB.
We design a reproducible, portable, and fair benchmark, simulating spatiotemporal workloads inspired by e-scooter trips in Berlin, which we simulate using developed tool for this thesis.
The benchmark evaluates bulk insert, simple read, and complex read scenarios using synthetic data and metrics such as latency, throughput, and success rate.
We deploy both systems on Kubernetes clusters on Azure Cloud using OpenTofu, varying cluster and client sizes.
Our findings reveal that CrateDB provides higher insert throughput, favoring high-ingest scenarios, while MobilityDBC shows robust support for complex spatiotemporal queries and maintains high success rates under analytical load.
We identified, that CrateDB is can perform simple spatiotemporal queries, yet is incapable of executing complex queries on large datasets.
Furthermore we see the limitations of our research, such as not benchmarking the UPDATE and DELETE queries.
Those limitations are viable directions for future research.

% GERMAN
\clearpage
\begin{otherlanguage}
  {ngerman}
  \section*{Kurzfassung}
  Die Nachfrage nach räumlich-zeitlichen Datenbanken steigt mit der Entstehung von Smart Cities, und große, horizontal skalierbare Systeme werden benötigt, um IoT-Workloads zu behandlen.
  In dieser Arbeit vergleichen wir die horizontale Skalierbarkeit zweier Open-Source, verteilter Datenbankmanagementsysteme (DDBMS): MobilityDB, erweitert mit Citus für horizontale Skalierbarkeit (MobilityDBC), und CrateDB.
  Wir entwickeln einen reproduzierbaren, portablen und fairen Benchmark, der räumlich-zeitliche Workloads simuliert, die von E-Scooter-Fahrten in Berlin inspiriert sind und die wir mit einem für diese Arbeit entwickelten Tool simulieren.
  Der Benchmark bewertet Szenarien mit INSERTS, einfachen und komplexen READS anhand synthetischer Daten und Metriken wie Latenz, Durchsatz und Erfolgsrate.
  Wir implementieren beide Systeme auf Kubernetes-Clustern in der Azure Cloud mit OpenTofu und variieren dabei Cluster- und Clientgrößen.
  Unsere Ergebnisse zeigen, dass CrateDB einen höheren Einfügedurchsatz bietet und damit Szenarien mit hohem Datenaufkommen besser passt, während MobilityDBC komplexe räumlich-zeitliche Abfragen zuverlässig unterstützt und hohe Erfolgsraten unter analytischer Belastung beibehält.
  Wir haben festgestellt, dass CrateDB zwar einfache räumlich-zeitliche Abfragen durchführen kann, jedoch nicht in der Lage ist, komplexe Abfragen auf großen Datensätzen auszuführen.
  Darüber hinaus erkennen wir die Grenzen unserer Forschung, beispielsweise das fehlende Benchmarking der UPDATE- und DELETE-Abfragen.
  Diese Einschränkungen bieten vielversprechende Ansätze für zukünftige Forschung.
\end{otherlanguage}

\clearpage

\tableofcontents

\section{Introduction}
\label{cha:introduction}

Spatio-temporal databases are increasing in popularity due to increased production of such data by smart cities, personal light electric vehicles and numerous IoT devices.
As the need for such databases raises, developers reach for production ready, scalable solutions.
One of spatiotemporal databases is MobilityDB, built on top of the spatial extension PostGIS, extends PostgreSQL's capabilities regarding spatio-temporal aspects~\parencite{zimanyiMobilityDBMobilityDatabase2020}.
Previous research has shown that MobilityDB can also scale horizontally using another extension, Citus~\parencite{bakliDistributedMobilityData2020, bakliDistributedMovingObject2019, cubukcuCitusDistributedPostgreSQL2021}.
In this thesis we will refer to distributed MobilityDB using Citus as \mobilitydbc.
As a comparison, CrateDB is a natively distributable SQL database designed for scalability, high performance, and real-time applications.
Both of the solutions are open source.

While the runtimes of specific queries of BerlinMOD benchmark by \textcite{duntgenBerlinMODBenchmarkMoving2009} have been evaluated on 4 and 28 node clusters of \mobilitydbc~by \parencite{bakliDistributedMobilityData2020}, its horizontal and vertical scalability still remains underexplored.
Previous research mostly focuses on designing queries for single node systems, and often does not compare systems between one another.
The other system, CrateDB, also has not been fully evaluated for its usability in spatiotemporal use cases.
Moreover, benchmarking with a wider range of cluster sizes is necessary to establish the scalability pattern.
Additionally, a comparison between \mobilitydbc and CrateDB in terms of horizontal scalability has not been conducted, which could provide valuable insights into their respective strengths and weaknesses.

We therefore make the following contributions in this thesis:
\begin{enumerate}
	\item We discuss the unique value propositions of both databases (\cref{cha:background})
	\item We design and implement a benchmark of common functionality of both databases that addresses the scalability according to requirements we have identified (\cref{cha:benchmarkdesign})
	\item We setup and run the experiment on 4 cluster sizes using our custom benchmark (\cref{cha:evaluation})
	\item We analyze the results for scalability patterns and discuss the results of the benchmark (\cref{cha:evaluation})
\end{enumerate}

\clearpage
\section{Background}
\label{cha:background}

Explain the concepts for someone who has background in cs but not in the exact field


% TODO:distributed systems and challenges there
Distributed systems have several challenges, such as network latency, data consistency, availability, partition tolerance (CAP theorem), and fault tolerance.
Both databases handle this challenges differently.
While CitusDB needs to implement its own mechanisms, MobilityDB passes those challenges to Citus extensions.
As this thesis purpose, I want to compare system of MobilityDB + Citus against CrateDB, by all future references of MobilityDB+C I mean MobilityDB + Citus
The MobilityDB approach allows the developers to focus on spatiotemporal aspects of the database.
This shows by the available complex queries in the system compared to the other solution.
CitusDB on the other hand focuses on the real-time aspect of the DB, which makes it a better solution for IoT devices and real-time analytics.


% TODO:spatiotemporal concepts and queries, explain different kinds, stationary and moving databases
postgresql
Maintainer: PostgreSQL Global Development Group
License: PostgreSQL License, which is a liberal, OSI-approved open source license similar to the BSD or MIT licenses

citus
Maintainer: Citus Data team, which is now part of Microsoft. It is developed as an open-source extension for PostgreSQL.
License: Follows postgresql license

mobilitydb
Maintainer: its own open-source community, and it is also an OSGeo community project.
License: Follows postgresql license


% TODO: small comparison of Mobilitydb and CrateDB (maybe a table)
\begin{table}[h]
  \centering
  \begin{tabular}{|c|c|c|}
    \hline
    Category & MobilityDB & CrateDB \\
    \hline
    Open source & Yes & Yes \\
    Maintained by & PostgreSQL by PostgreSQL Global Development Group, Citus by Citus Data team, which is now part of Microsoft, MobilityDB by its own open-source community, and it is also an OSGeo community project& Crate.io \\ 
    Supports SQL Queries & Yes & Yes \\
    Built from ground up & No (it is built on PostGIS and Citus) & Yes \\
    License & OSI-approved open source license similar to the BSD or MIT licenses & Apache 2.0 \\
    \hline
  \end{tabular}
  \caption{Sample Table}
\end{table}

\clearpage
\section{Benchmark Design}
\label{cha:benchmarkdesign}

In this thesis we compare two DDBMSs, CrateDB and \mobilitydbc to assess their scalability in spatial-temporal workloads typical for IoT usecases.
Spatio-temporal aspects are our primary target as the value proposition of \mobilitydbc is exactly that.
Our goal is not to advocate for one system over the other but to investigate their respective strengths and limitation in processing ST data at scale.
To evaluate the scalability we have designed a benchmark seen in Figure~\ref{fig:benchmark_design}.

\begin{figure}[ht]
	\centering
	\begin{tikzpicture}[
			font=\small,
			node distance=0.5cm,
			box/.style={
					rectangle,
					draw,
					text width=2.5cm,
					minimum height=1cm,
					align=center
				},
			client/.style={
					box,
					fill=blue!10,
					draw=blue!70
				},
			coordinator/.style={
					box,
					fill=orange!10,
					draw=orange!70,
				},
			worker/.style={
					box,
					fill=purple!10,
					draw=purple!70
				},
			group/.style={
					rectangle,
					draw,
					dashed,
					inner sep=0.2cm
				},
			arrow/.style={
					->,
					>=latex
				}
		]
		% Client components
		\node[client] (db_client) {\textbf{DB Client}};
		\node[client, below=of db_client] (datagen) {Data Generator};
		% Client connections
		\draw[arrow] (datagen) -- (db_client);
		% Coordinator node
		\node[coordinator, right=5cm of db_client] (cn) {
			\textbf{Master Node}};
		% Worker nodes
		\node[worker, below=of cn] (w2) {
			\textbf{Worker Node \\ 2}};
		\node[worker, left=of w2] (w1) {
			\textbf{Worker Node \\ 1}};
		\node[worker, right=of w2] (w3) {
			\textbf{Worker Node \\ N}};
		% Connections of coordinator to workers
		\draw[arrow] (cn) -- (w1);
		\draw[arrow] (cn) -- (w2);
		\draw[arrow] (cn) -- (w3);
		% Client to SUT connections
		\draw[arrow] (db_client) to[bend left=13] node[above] {Queries/Writes} (cn);
		\draw[arrow] (cn) to[bend left=5] node[below] {Results} (db_client);

		% Group boxes
		\begin{pgfonlayer}{background}
			% Benchmarking client
			\node[group, fit=(db_client) (datagen)] (client) {};
			\node[above] at (client.north) {Load generator};

			% System Under test
			\node[group, fit=(cn) (w1) (w2) (w3)] (sut) {};
			\node[above] at (sut.north) {System Under Test};
		\end{pgfonlayer}
	\end{tikzpicture}
	\caption{
		Load generator will write generated spatiotemporal data and execute
		the workload against System Under Test (MobilityDB/CrateDB cluster).
	}
	\label{fig:benchmark_design}
\end{figure}

In the benchmark the load generator will connect with the System Under Test (SUT) and issue queries.
It will log metrics allowing later analysis of the findings.
The System Under Test, will be run on the same resources i.e., MobilityDB will be deployed on the same resources as CrateDB.
The resources for the SUT will be restarted between each benchmark run.

% Things to explain:
% * this benchmark is designed for cloud
\subsection{Infrastructure}
We design this benchmark to be deployed and run on the cloud for two reasons.
Firstly, we did not have access to multiple local machines with the same specification to ensure no bottlenecks.
Secondly, the repeatability and relevance of such approach is improved, as the benchmark can be replicated by other parties without them having access to a collection of devices and cloud deployed systems are a common (CITATION).

% * this benchmark uses infrastructure as code to allow reproducibility
% * this benchmark each ddbms is run on a Kubernetes cluster
To improve reproducibility and repeatability all parts of the benchmark are documented in code, this includes the resources deploy on the cloud.
Using infrastructure as code allows multiple benchmark runs with the exact same configuration without option of skipping checking in some setting on a cloud provider web admin console.

% * explain that both ddbms will be run on the same resources (same vm sizes)
Using cloud and infrastructure as code allows us to deploy both CrateDB and MobilityDB on the exact same resources with the same configuration of them i.e., same virtual machine types/sizes.
This allows us to be fair towards both systems and provide them with the same resources.

\subsection{DDBMS cluster}
Documenting the DDBMS configuration through Kubernetes deployment files ensures that the software will be deployed with the same settings and versions using versioned container images.
Additionally by using Kubernetes, the benchmark can also be more easily developed on a local systems by using contenerization.
The local development has been an important consideration as the recent withdraw of google from issuing cloud tokens to our university, lead to uncertainty to where the benchmark will be deployed.
Kubernetes has been chosen over using bash scripts for installing DDBMS on multiple virtual machines for three reasons.
(1) It allows easier deployment of a DDBMS cluster, which is beneficial when deploying multiple cluster sizes and resetting their state by destroying them.
(2) The opportunity to experiment with a new technology also played a role, as we were unfamiliar with working with Kubernetes before.
(3) Relevancy, as deployment of such clustered DDBMS systems is often done this way in production environments (CITATION HERE)


\subsection{Load generator}
% * this benchmark uses synthetic workoad load generator
We decided to split the benchmark into two major components, the DDBMS cluster and a load generator.
Load generator will be a software simulating multiple clients on a DDBMS cluster by using multiple threads.
It will use a closed synthetic workload generation with fixed user pool with requests with random parameters generated on the fly based on query templates.
By using a synthetic workload we sacrifice the realism of the application by improving understandability and simplifying the development of the benchmark.
This approach allows easy data collection, compared to using multiple load generators running on multiple hosts and makes the benchmark cheaper to run.
We believe this strategy is viable, as the load generator primarily issues requests, which are i/o heavy, so the performance of the computer running the software shouldn't be a bottleneck.
Simulating multiple clients at the same time strives to mimic the IOT workload where there are many devices issuing requests at the same time.
% * this benchmark load generator is placed on the same VPC in the cloud as the cluster with running DDBMS
The load generator is also put on the cloud in the same virtual private cloud (VPC) as the DDBMS cluster.
Putting the load generator next to the SUT instead of running it on our local computer allows us to minimize latency and potential disturbances while the benchmark runs i.e., packages dropped due to network problems.


\subsection{Workloads and metrics}
% * explain which metrics have been chosen and why:
%   - latency
%   - throughput
%   - success rate (to see whether the system answers all queries and doesn't drop them)
During the benchmark run, the load generator will log the following metrics for the analysis afterwards:
(1) Latency in milliseconds between sending the request to DDBMS and receiving a response.
(2) Whether the request/query has been resolved successfully or not.
This allows us to avoid a situation where one SUT would perform better by rejecting multiple queries.
(3) Throughput, by logging how many queries have been sent and the total time it took to do it, we will calculate the throughput.
We chose this metric as it is an important in IOT devices as they are commonly deployed in great amounts and send a lot of data i.e., shareable e-scooters in a city which report their location and status.


% * explain which benchmark modes? will be tested and why 
%   - inserts - very realistic for iot usage where a lot of devices share their state/location
We have identified three scenarios for the SUT which we will test.
(1) Data insertion, a scenario where the clients will be sending data to insert to the DDBMS.
By this we want to benchmark how the SUT will perform when multiple clients try insert data i.e., e-scooters deployed in a city reporting their status and location.
Here we find the write throughput metric extremely relevant. (CITY SOME SOURCE)
% * explain which benchmark modes? will be tested and why 
%   - simple queries - important for real time usecases, like users of an app (finding the closest e-scooter)
(2) Simple read queries, in this scenario we try to simulate queries done by the users of a system i.e., people using the e-scooter app to find the closest e-scooters.
In this scenario we benchmark the read throughput, as it is a relevant metric for the clients of the system.
% * explain which benchmark modes? will be tested and why 
%   - complex queries - important for data analysis, to find out patterns and optimizations (e.g.,used by the e-scooter sharing company)
(3) Complex read queries, here we try to simulate a scenario of a complex queries used for analysis and optimizations i.e., queries done by the e-scooter company to find patterns in the traffic to optimize the placement of the e-scooter stations throughout a city or to gain insight into their data.

\subsection{Scalability}
% NOTE: multiple sizes of the cluster will be used (2,3,4,5) to establish scalability pattern
To establish a scalability pattern multiple sizes of a cluster will be run to evaluate the horizontal scalability.
For each cluster size we run a benchmark of \mobilitydbc and CrateDB.

% NOTE: different client amounts, different number of simultaneous connections simulated by the load generator (100, 1000, 10000)
Not only we will test different cluster sizes, but we will also test a different amount of simultanous client connections.
The load generator will be able to simulate varying number of them using threads i.e., 1000 and 10000 connections.
Multiple configurations of amount of clients allow us to check whether the performance of the queries is improved, such as the response time, or the amount of simultaneous connections handled.
This could lead to important findings as some companies may value the performance of the queries for their data analysis more important compared to the simultaneous connections handled.
The latter one may be more important for a company where vast amount of IoT devices connect to the DDBMS at the same time.


\clearpage
\section{Evaluation}
\label{cha:evaluation}

\subsection{Implementation}
\label{sec:implementation}

\subsubsection{Dataset Generator}
We have decided to not use an off the shelf dataset or dataset generator like [INSERT BERLIN MOD CITATION], but to implement our own.

We developed a data generator simulating e-scooter trips in Berlin using Python 3. \footnote{\url{github.com/erykksc/escooter-trips-generator}}
We used the OpenStreetMap data through OSMnx library to extract the biking network, points of interest and administrative boundaries (we call them localities).
Inside the repository we also define a flake.nix file, which allow to match the exact versions of Python interpreter and all the libraries we used, ensuring repeatability and reproducibility.
Additionally we use a seed value in order for the pseudo random generators to yield the same results each run.


We have simulated the 1 billion trips from 01-01-2020 to 31-12-2025 in a following manner:
\begin{enumerate}
\item Start point and start time of every ride is chosen randomly from uniform distribution
\item A random bearing is chosen, also from a uniform distribution
\item The route traverses through nodes in a direction of a random bearing, choosing nodes with the smallest bearing difference to our chosen bearing
\item The speed between the nodes along the trip is modeled with a uniform distribution between 10-20km/h (as 20km/h is max legal speed in Germany and a driver needs to slow down on turns or traffic lights)
\item Using the speed and the distance between the nodes, the time to travel between the nodes is computed
\item Using the route, start time, and time to travel between individual nodes of the route, a single trip is created.
	A trip consist of many events. 
	An event is a single data entry with following fields and types:
	\begin{itemize}
		\item \textbf{event\_id} - UUIDv4
		\item \textbf{trip\_id} - UUIDv4
		\item \textbf{timestamp} - ISO 8601 timestamp with timezone offset
		\item \textbf{latitude} - float
		\item \textbf{longitude} - float
	\end{itemize}
\end{enumerate}

We saved the simulated trips into a CSV file for our load generator to consume.

\subsubsection{Load Generator}
We implemented the load generator in golang v1.24.
We choose this programming language due to its builtin concurrency features, go-routines, which are a lightweight, user-space threads managed by the Go runtime, and channels, which allow safely communication between go-routines.
We made this decision in order to simulate many simultaneous DBMS clients using as possibly little memory in order to simulate many.
This is possible because go-routine threads are managed by go-runtime and not OS, the memory footprint of them is small.

We implemented 3 modes to the load generator:
\begin{enumerate}
	\item \textbf{initialize} - initializes tables and indexes
	\item \textbf{insert} - inserts e-scooter events into DBMS
	\item \textbf{query} - queries DBMS using randomized queries from a template
\end{enumerate}


\subsubsection{Insert Queries}
We insert the e-scooter events from the generated CSV file into DBMS using multiple concurrent connections.

The insert process works on a basis of a worker queue [CITATION] i.e., main thread reads data from the CSV file and adds jobs to the queue.
The workers read the jobs from the queue and execute insert queries on the DBMS while collecting metrics.

The queue is implemented using go-channel avoiding race conditions and manual mutex locks as the go-channels provide built-in synchronization.
Individual worker threads measure the metrics from the DBMS alongside the time they have spent waiting for their next job.
We have added this mechanism in order to ensure that the speed at which main thread reads from file is not a bottleneck as we want to ensure that the load-generator is not a bottleneck in the benchmark.

\subsubsection{Read Queries}

We split the read queries into two categories, simple and complex.
By doing that we want to simulate two user categories (1) clients (2) analysts.

The simple queries, the ones simulating individual clients, consist of queries that query for information about a single trip.
We have identified 8 queries relevant for an e-scooter user:
\begin{enumerate}
	\item \textbf{Length of a trip} - retrieve the length of a specific trip e.g., in the summary after a trip
	\item \textbf{Trip events} - retrieve all events of a trip e.g., in order to display them on a map
	\item \textbf{Average speed of a trip} - retrieve average speed during the trip e.g., in the summary after a trip
	\item \textbf{Trip start locality} - retrieve the locality the trip started in i.e., part of the city where the trip started
	\item \textbf{Trip finish locality} - retrieve the locality the trip ended in i.e., part of the city where the trip ended
	\item \textbf{First and last location of a trip} - retrieve the first and last location of a trip e.g., in order to show simplified summary view on a map
	\item \textbf{POIs close to the finish location} - retrieve points of interest in a specific radius around the finish location e.g., in order to show shops nearby
	\item \textbf{POIs within radius during the trip} - retrieve which points of interests the user has driven past through during his trip e.g., in order to mark certain touristic attractions as visited
\end{enumerate}

Besides the simple queries, we have defined 7 complex queries.
Those queries aim to mimic the queries an analyst may sent to DMBS in order to gain insight into traffic, find possible optimization points.
In order to imitate that we defined the following queries, each of them queries about a specific time frame:
\begin{enumerate}
	\item \textbf{Trips in locality} - Which trips have passed through a specified locality in time frame
	\item \textbf{Most visited POIs} - which points of interests had trips that were in specified distance from them in a specified time frame
	\item \textbf{N closest trips to POI} - find n trips that came the closest to a point of interest in a specified time frame
	\item \textbf{Average trip duration per locality} - calculate the average trip duration per locality in a specified time frame
	\item \textbf{Start and end in different localities} - how many trips started and ended in different locality in a specified time frame
	\item \textbf{Event density heat-map per locality} - generate heat-map of e-scooter usage by area and hour
	\item \textbf{Trips summaries} - summaries about trips in time frame e.g., minimal, maximal, average of length, duration of a trip as well as of the event count for trips
\end{enumerate}

% TODO: add the section that mobilitydb is inspired from the tutorial and then the cratedb has been 
% created to mimic this functionality. There were
The queries are inspired by the online tutorial [FOOTNOTE LINK], but have been adapted fo

All the queries are parametrized so they do not repeat.
We generate the actual queries on the fly inside the load generator during execution. 
The random generator is set to a seed and we use a worker queue pattern, same as in the insert mode.
As the seed is set based on the query number and base seed, allowing reproducibility of every query.
We choose a query template in a rotating fashion, so that the each subsequent template is different from the past one.
This ensures that we will not be biased towards one DBMS by generating different queries for it.

\subsubsection{Infrastructure}
We implemented the infrastructure as code using OpenTofu, an open source terraform fork.
Using it we defined three deployments:
\begin{enumerate}
	\item \textbf{Shared} - defines shared infrastructure between load-generator and AKS-cluster i.e., subnet, virtual network, and a resource group
	\item \textbf{Load Generator} - defines the virtual machine (VM) used as load-generator
	\item \textbf{AKS Cluster} - defines Kubernetes cluster running on Azure Kubernetes Service (AKS)
\end{enumerate}

Using three deployments allows us to take down the Kubernetes cluster without affecting the load generator VM.
This is important as the load generator stays the same between benchmark runs and the size of the AKS-cluster changes.

\subsubsection{Deployment}

We have decided to use containerization in order to improve portability and allow easy local development.
We automated the deployment by choosing Kubernetes as the container orchestration tool.
We chose it over Docker swarm as it has wider adoption, thus improving relevancy as it is more likely that DBMS will be deployed this way.

We used the provided official Kubernetes operator to deploy the CrateDB clusters.

For the MobilityDBC there was no official nor unofficial docker image, so we developed our own.
\footnote{\url{https://github.com/erykksc/citus-mobilitydb}}
We used PostgreSQL 17 image as base for the image as it is the newest supported version by Citus at time of writing.
\footnote{\url{https://www.citusdata.com/updates/v13-0/}}
As the PostgreSQL 17 image is based on Debian, we used the official instructions from Citus webpage on top of PostgreSQL.
\footnote{\url{https://www.citusdata.com/download}} to install Citus 13 (newest at time of writing)
The installation of Citus required adding additional repository to Debian's package manager APT, while MobilityDB package was in the official repositories.
We automatically enable the Citus, MobilityDB, and PostGIS (dependency of MobilityDB) on the container startup.
We published the image to GitHub Container Registry (GHCR) under MIT License.

Individual worker nodes of MobilityDBC need to be manually connected to the coordinator node, as Citus doesn't support auto discovery.
For this job, we developed a Manager, a Docker container which when added to the Kubernetes cluster with right permissions, automatically detects new pods running Citus and connects them to the Coordinator.
We also published the images to GHCR under MIT License.
\footnote{\url{https://github.com/erykksc/citus-k8s-manager}}
CitusDB clusters do not require such manager as the nodes support auto discovery and elect the master node themselves, provided a valid configuration upfront.[FOOTNOTE]

We created two Helm charts in order to simplify the deployment of the DBMS, one for CrateDB and one for MobilityDB.
\footnote{\url{[GITHUB REPO URL]}}
Using Helm charts we are able to deploy a collection of Kubernetes deployments such as stateful sets with the nodes, internal and external services, storage classes required to run DBMS.
We choose Helm charts as they allow parametrization using templating, so change of Storage Class or number of nodes in cluster can be changed using a CLI argument.
Kubernetes by itself does not support this feature and relies on external tools, such as Helm.
Helm is the official Kubernetes package manager and is the most commonly used one [CITATION].

\subsection{Experiment}
\label{sec:experiment}

% \subsection{Load Generator}
% This approach allows easy data collection, compared to using multiple load generators running on multiple hosts and makes the benchmark cheaper to run.
% We believe this strategy is viable, as the load generator primarily issues requests, which are i/o heavy, so the performance of the computer running the software shouldn't be a bottleneck.
% % * this benchmark load generator is placed on the same VPC in the cloud as the cluster with running DDBMS
% The load generator is also put on the cloud in the same virtual private cloud (VPC) as the DDBMS cluster.
% Putting the load generator next to the SUT instead of running it on our local computer allows us to minimize latency and potential disturbances while the benchmark runs i.e., packages dropped due to network problems.


% \subsection{DDBMS cluster}
% We document the DDBMS cluster configuration through Kubernetes deployment files ensuring that the software will be deployed with the same settings and versions using versioned container images.
% Additionally by using Kubernetes, the benchmark can also be more easily developed on local systems by using contenerization.
% The local development has been an important consideration as the recent withdrawal of Google from issuing cloud tokens to our university, lead to uncertainty to where the benchmark will be deployed.
% Kubernetes has been chosen over using bash scripts for installing DDBMS on multiple virtual machines for two reasons.
% (1) It allows easier deployment of a DDBMS cluster, which is beneficial when deploying multiple cluster sizes and resetting their state by destroying them.
% (2) Relevancy, as deployment of such clustered DDBMS systems is often done this way in production environments (CITATION HERE)

% In the benchmark the load generator will connect with the System Under Test (SUT) and issue queries.
% It will log metrics allowing later analysis of the findings.
% The System Under Test, will be run on the same resources i.e., MobilityDB will be deployed on the same resources as CrateDB.
% The resources for the SUT will be restarted between each benchmark run to ensure a clean state.

% \subsection{Design Objectives}
% The benchmark should be relevant to real-world use cases, especially spatial-temporal workloads as this is the primary value proposition of MobilityDB over other databases.
% We accomplish that by modelling queries after common use cases like, time slice queries, window queries, and spatiotemporal joins.
% We deploy the SUTs on a Kubernetes cluster in Microsoft Azure, resembling common deployment method of companies.
% Such deployment using infrastructure as code, allows us to share the configuration of SUTs, as well as information of provisioned cloud resources, allowing reproducibility of our results.

% The decision to choose Kubernetes over shell scripts or other automation tools such as Ansible has reduced understandability (as not everybody is familiar with container orchestration tools such as Kubernetes), but allowed us to improve portability, relevance and simplified development.
% This trade off has been made as unexpectedly because of the cloud infrastructure provider, where we wanted to run the benchmark on initially, cut off credits for the university right before the start of the thesis writing.
% Such change, made us switch to a solution that was more portable.
% Using containerization, allowed us to develop on single host machine without creating VMs and also made it possible to run our benchmark on other cloud providers or university servers/Kubernetes cluster.
% Furthermore, use of containers improves repeatability as the node will have exact same versions of the software installed on them.

% The SUTs are benchmarked using equivalent conditions ensuring fairness by
% (1) deploying them on the same cloud resources (processor types, memory, storage, and networking),
% (2) running semantically analogous queries,
% (3) connecting same amount of clients to them, and by
% (3) inserting same amount of data.
% Notably, \mobilitydbc~supports more complex queries than CrateDB and because of that we will benchmark queries supported by both systems.
% We suspect that \mobilitydbc~ being full ACID compliant will result in reduced raw performance, but in this thesis we focus on scalability patterns so are making this trade-off to fairness.

% Our benchmark evaluates scalability by measuring how each SUT performs under growing cluster sizes and data volumes.

% We decided to benchmark the following sizes for the SUT clusters, 2, 3, 4, and 5.
% Preferably we would benchmark the DDBMS on larger cluster sizes to improve relevance of our findings, but due to resource constrain we choose to do it on a set of small ones to reduce costs.
% Despite the small difference in size between the sizes, we hope to see a pattern which would allow us to create conclusions about scalability.

% To check how well the databases handle multiple simultaneous connections and try to find a pattern, we have decided to benchmark the SUT on following number of simultaneous clients 100, 1000, 10000.

% % \subsection{Kubernetes explanation}
% %
% % Important aspect of this benchmark was the portability as there was uncertainty where would the experiment be conducted.
% % Three environments had to be taken into account.
% % First, the cloud, initially it was planned to run the experiment on Google Cloud Platform but unfortunately, Google suddenly stopped providing credits for the University.
% % Two, the server cluster of the University, with support for Kubernetes.
% % Our machines, for testing and in case it wouldn't be possible to use different computers.
% %
% % Ultimately we have decided to go with Kubernetes, as it provided a way to run the database systems on all of the three environments, as well as increasing the benchmark relevance, as we assume the production systems are likely gonna be deployed in such environment.
% % Additionally use of Kubernetes increased ease of use, as it is a common solution for distributed systems, along with robustness, as Kubernetes automatically revives dead nodes, allows to easily change replica count and extend the configuration for real life systems.
% %
% % For the data used for the benchmark we decided to use a synthetic data generator that would be run as a pod on the same Kubernetes cluster in order to minimize possible, latency and unknown variables on the network.
% \subsection{Quality Metrics}
% % TODO: Move this entire section to Evaluation chapter - metrics definitions, percentile breakdowns, IoT relevance justification
% To measure scalability we plot the Metric curves over increasing node counts.
% We choose the following metrics for our benchmark
% (1) Latency, it will be measured in milliseconds for both inserts and queries. We will create percentile breakdown (P50,P95, P99) to provide insight into tail latencies.
% (2) Throughput, we will measure the number of records written or queries completed per second.

% In the context of IoT devices, write throughput is important for the devices themselves, and the read latency is important for the users.
% Because of this, the benchmark results should be relevant.

% \subsection{Workload design}
% % TODO: Move to separate "Workload Design" chapter or merge into Evaluation. Move Go implementation details to Implementation chapter.
% We adopt a synthetic workload combined with a micro-benchmark approach where each run isolates one quality of system behavior (e.g., write latency, read throughput).
% This avoids confounding metrics and ensures interpretable results. We have developed a custom synthetic workload generator, written in Golang, which will issue concurrent requests using lightweight goroutines to simulate a fixed large number of clients.
% Golang has been chosen due to goroutines, which are lightweight threads (~2KB stack vs ~1MB stack when using OS threads) managed by the Go runtime, allowing us to simulate a large number of concurrent clients on a relatively small system.

% For the write workload we will simulate IoT devices emitting time-stamped spatial data of %TODO: what will the generated data be of?
% We will test batch and single-record inserts at varying ingestion rates to measure write throughput and latency.

% Whereas for the read workload, we will simulate %TODO: what will we simulate exactly?
% This will be performed by running following queries
% (1) Temporal range queries (e.g., last hour of data per device).
% (2) Spatial bounding box queries (e.g., devices within a bounding box).
% (3) Spatio-temporal window queries (e.g., average speed over 10-minute intervals).

% \subsection{Measurement}
% % TODO: Move this entire section to Evaluation chapter - measurement methodology, goroutine details, resource monitoring specifics
% We combined the load generator with the testing client in order for the same goroutine to issues requests as well as to measure their metrics.
% This way we keep the measuring simple and understandable and don't overcomplicate the setup.

% In order to minimize unknown variables, such as network latency, reliability, and interference of other devices, we put the load generator on the same Kubernetes cluster.
% The load generator pod runs on a separate Kubernetes node, a different host machine, isolated from database nodes to prevent influencing SUT performance.
% During the evaluation we monitor the resource usage of every node, focusing on the load generator in order to prevent it from becoming a bottleneck in the benchmark.

\clearpage
section{Discussion}
\label{cha:discussion}

\clearpage
\section{Related Work}
\label{cha:relatedwork}

The Brewer's \cite{brewerRobustDistributedSystems2000} CAP theorem and Abadi's \cite{abadiConsistencyTradeoffsModern2012} PACELC extension remains as the basis for design and evaluation of all horizontally scalable systems.
We demonstrated two DDBMS, making different trade offs when it comes to choosing latency (CrateDB) over consistency (MobilityDBC).

Bakli et. al.\parencite{bakliDistributedMobilityData2020, bakliDistributedMovingObject2019, cubukcuCitusDistributedPostgreSQL2021}



\clearpage
\section{Conclusion}
\label{cha:conclusion}

Data driven future requires databases capable of specialised workloads such as spatiotemporal ones.
In space of distributed databases CrateDB and MobilityDBC stand as valid choices.
Both of the DDBMS proved to be capable of horizontal scaling.

What's more, we identified the use-cases for each system.
We displayed that MobilityDBC is capable of performing complex spatiotemporal queries which CrateDB struggled with or was not able to perform them.
On the other hand, we demonstrated that CrateDB was capable of higher write throughput, making it ideal for high ingest workloads where only simple ST querying capacity is needed.

Concluding, the thesis topic and comparison of two DDBMS systems, CrateDB and MobilityDBC, proved to have depth.
Our findings showcase just a small part of the comparison of those two systems.
In the discussion section, we laid out all the limitations and possible future research possibilities, one of which is to benchmark performance of other cluster sizes, benchmark vertical scalability and benchmark different query types such as \verb|UPDATE| and \verb|DELETE|.



\newpage
\printbibliography
\end{document}
