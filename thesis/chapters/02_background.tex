\section{Background}
\label{cha:background}

Explain the concepts for someone who has background in cs but not in the exact field


% TODO:distributed systems and challenges there
Distributed systems have several challenges, such as network latency, data consistency, availability, partition tolerance (CAP theorem), and fault tolerance.
Both databases handle this challenges differently.
While CitusDB needs to implement its own mechanisms, MobilityDB passes those challenges to Citus extensions.
As this thesis purpose, I want to compare system of MobilityDB + Citus against CrateDB, by all future references of MobilityDB+C I mean MobilityDB + Citus
The MobilityDB approach allows the developers to focus on spatiotemporal aspects of the database.
This shows by the available complex queries in the system compared to the other solution.
CitusDB on the other hand focuses on the real-time aspect of the DB, which makes it a better solution for IoT devices and real-time analytics.


% TODO:spatiotemporal concepts and queries, explain different kinds, stationary and moving databases
postgresql
Maintainer: PostgreSQL Global Development Group
License: PostgreSQL License, which is a liberal, OSI-approved open source license similar to the BSD or MIT licenses

citus
Maintainer: Citus Data team, which is now part of Microsoft. It is developed as an open-source extension for PostgreSQL.
License: Follows postgresql license

mobilitydb
Maintainer: its own open-source community, and it is also an OSGeo community project.
License: Follows postgresql license


% TODO: small comparison of Mobilitydb and CrateDB (maybe a table)
\begin{table}[h]
  \centering
  \begin{tabular}{|c|c|c|}
    \hline
    Category & MobilityDB & CrateDB \\
    \hline
    Open source & Yes & Yes \\
    Maintained by & PostgreSQL by PostgreSQL Global Development Group, Citus by Citus Data team, which is now part of Microsoft, MobilityDB by its own open-source community, and it is also an OSGeo community project& Crate.io \\ 
    Supports SQL Queries & Yes & Yes \\
    Built from ground up & No (it is built on PostGIS and Citus) & Yes \\
    License & OSI-approved open source license similar to the BSD or MIT licenses & Apache 2.0 \\
    \hline
  \end{tabular}
  \caption{Sample Table}
\end{table}
