\section{Conclusion}
\label{cha:conclusion}

Data driven future requires databases capable of specialised workloads such as spatiotemporal ones.
In space of distributed databases CrateDB and MobilityDBC stand as valid choices.
Both of the DDBMS proved to be capable of horizontal scaling.

What's more, we identified the use-cases for each system.
We displayed that MobilityDBC is capable of performing complex spatiotemporal queries which CrateDB struggled with or was not able to perform them.
On the other hand, we demonstrated that CrateDB was capable of higher write throughput, making it ideal for high ingest workloads where only simple ST querying capacity is needed.

Concluding, the thesis topic and comparison of two DDBMS systems, CrateDB and MobilityDBC, proved to have depth.
Our findings showcase just a small part of the comparison of those two systems.
In the discussion section, we laid out all the limitations and possible future research possibilities, one of which is to benchmark performance of other cluster sizes, benchmark vertical scalability and benchmark different query types such as \verb|UPDATE| and \verb|DELETE|.

