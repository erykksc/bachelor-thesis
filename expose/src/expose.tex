\documentclass{article}

% Language setting
\usepackage[english]{babel}

% Set page size and margins
\usepackage[a4paper,top=2cm,bottom=2cm,left=3cm,right=3cm,marginparwidth=1.75cm]{geometry}

% Useful packages
\usepackage{amsmath}
\usepackage{graphicx}
\usepackage[colorlinks=true, allcolors=blue]{hyperref}

% For inline enumerations
\usepackage{enumitem}

\title{Evaluating the Vertical and Horizontal Read and Write Scalability of MobilityDB \\[1ex] \large Bachelor's Thesis Exposé}
\author{Eryk Karol Kściuczyk}
\date{} % This removes the date

\begin{document}
\maketitle


% Context
Spatio-temporal databases are increasing in popularity due to increased production of such data by numerous IoT devices.
As the need for such databases increases, developers try to reach for open source and familiar solutions.
MobilityDB, built on top of the spatial extension PostGIS, extends PostgreSQL's capabilities regarding spatio-temporal aspects.
Previous research has shown that the MobilityDB extension can be partially used in conjunction with Citus,
a PosgreSQL extension, which provides horizontal scalability
options.\footnote{
Bakli, Sakr and Zimanyi, "MobilityDB: A Mobility Database Based on PostgreSQL and PostGIS," ACM Transactions on Database Systems (TODS), Volume 45, Issue 4, Article No. 19, Pages 1-42, December 6, 2020
} It does it by sharding data across multiple nodes while retaining the rich features of PostgreSQL.
While specific queries and their runtimes have been evaluated\footnote{
Bakli, Sakr, and Zimanyi, "Distributed Mobility Data Management in MobilityDB," Proceedings of the 2020 21st IEEE International Conference on Mobile Data Management (MDM), June 2020
}, the vertical and horizontal scalability of the platform has not been fully explored and measured.

% Problem Statement
MobilityDB's distributed capabilities, enabled by integration with Citus, claim to enhance scalability.
% TODO: try to shorten this section into one or two sentences mentioning the unexplored angles
However, the platform's ability to efficiently scale read and write operations under real-world workloads has not been rigorously evaluated.
Current research has primarily focused on optimizing read query performance,
such as complex spatial and spatio-temporal queries. However, write throughput—critical for handling high-ingest
workloads typical of IoT applications—remains underexplored, especially in scenarios involving concurrent writes
across distributed nodes.
Questions remain about latency, throughput, and scalability in vertical and horizontal configurations when handling realistic spatio-temporal datasets.

% Solution Approach
In this thesis we aim to benchmark the vertical and horizontal scalability regarding both read and write
operations with generated spatio-temporal data to measure metrics like request latency and write throughput.
% TODO: shorten by merging it with the next sections that also mention it
We plan to deploy MobilityDB on single-node and multi-node setups.
Additionally, we will analyze how query options affect impact on the performance in different configurations.
Afterwards, we want to compare scalability trade-offs between vertical and horizontal approaches.

% Aspired Implementation
Benchmarking data will be generated using a custom data generation tool specifically developed for this study.
The tool will simulate the movement and locations of birds, chosen due to their seasonal migration patterns,
localized movements within standard habitats, varying flight heights and times.
% TODO: To musch detail for an expose, (remove/shorten)
Additionally, modeling different bird species will enable simulation of diverse flight altitudes, fly times and
migration times, providing a robust dataset for spatio-temporal analysis.
%
We will compare the performance of MobilityDB combined with Citus on different hardware configurations using 
Google Compute Engine service on Google Cloud Platform.
We will evaluate several scenarios, including vertically scaling a single node and scaling out by distributing MobilityDB across several instances.

\begin{figure}[ht]
    \centering
    \includegraphics[width=0.8\textwidth]{./resources/benchmark-architecture.png}
    \caption{
	Architecture of the benchmarking setup. 
	The benchmarking client generates synthetic bird movement data and executes the workload against 
	the distributed MobilityDB cluster. 
	The System Under Test consists of a coordinator node that distributes queries across multiple worker nodes,
	each running PostgreSQL with MobilityDB and Citus extensions. 
	Metrics are collected to analyze system performance and scalability.
    }
    \label{fig:image}
\end{figure}

% Analysis/ Evaluation
% TODO:
% - Make it shorter
% - Shrink the diagram
For evaluation of our results, we will employ multiple statistical and visualization approaches to ensure
comprehensive analysis of the system's performance characteristics. Query latencies and throughput measurements
will be analyzed using confidence intervals (95\% CI) to account for performance variability in distributed systems. 
To visualize the distribution of latencies across different scaling configurations, we will utilize Empirical
Cumulative Distribution Function (ECDF) plots, to reveal performance patterns and tail latencies.
For write throughput analysis, we will present time series plots showing sustained throughput over extended
periods, accompanied by statistical summaries including median, 95th, and 99th percentiles.
In order to validate the statistical significance of performance differences between vertical and horizontal scaling
approaches, we will conduct paired t-tests where appropriate.
Each benchmark configuration will be run multiple times with sufficient duration (minimum 30 minutes per run)
to ensure statistical reliability and to capture any performance degradation or variance over time.
\end{document}
